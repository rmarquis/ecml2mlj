{\bf Skill rating} dates at least as far back as the Elo system
\cite{elo78TheRatingOfChessPlayers}, the idea of which is to model the
probability of the possible game outcome as a function of the two
players' skill levels. Players' skill levels are updated after each
game in a way such that the observed game outcome becomes more likely
and the summation of players' ratings remains unchanged.

The Elo system cannot handle the case when three or more teams
participate in one match, a disadvantage addressed by TrueSkill
\cite{herbrich06569}. %Compared with the Elo system, TrueSkill models each
%player's skill by a Gaussian distribution that simply requires
%maintaining its mean and variance and updates these parameters based on
%WLD match outcomes.
Further extensions of TrueSkill incorporate time-dependent
skill modeling for historical data~\cite{dangauthier07337}.

%TrueSkill assumes that skill update just involves those teams
%participating in a match, i.e., other players' skills are not
%effected. To infer entire time series of skills of players, the
%authors in \cite{dangauthier07337} have extended TrueSkill by
%smoothing through time instead of filtering. Note that both consider
%that players' skills are independent.

In \cite{birlutiu07ExpectationPropagation}, the authors model and learn the
correlation between all players' skills when updating skill beliefs, and
develop a method called ``EP-Correlated", contrasted with the
independent assumption on players' skills
(EP-Independent).  Empirically, EP-Correlated outperforms
EP-Independent on professional tennis match results;
this suggests modeling correlations in extensions of the
score-based learning presented here.

These skill learning methods all share a common feature that they are
restricted to model WLD only and have to discard meaningful
information carried with scores.  While we proposed score-based extensions
of TrueSkill in this work; it remains to incorporate other extensions
motivated by this related work.

{\bf Score modeling} has been studied since the 1950s
\cite{Moroney56FactsFromFigures}
\cite{dixon97ModellingAssociationFootball} \cite{Glickman98JASA}
\cite{Karlis03AnalysisOfSportsData}
\cite{karlis09BayesianModellingFootballOutcomes}; one of the most
popular score models is the Poisson model, first presented in
\cite{Moroney56FactsFromFigures}, and this work continues to the
present~\cite{karlis09BayesianModellingFootballOutcomes}. Other
commonly used score models are based on normal distributions
\cite{Glickman98JASA}. However, it appears that most score-based
models do not distinguish offence and defence skills of each team and
the results here indicate that such separate offence/defence skill
models can perform better than univariate models with limited data.

More recently, \cite{Baio:10JAS} introduced a log-linear random effect model to model the number of goals for a football match, which takes nto account home field advantages and distinguish teams' attack and defense skills, and proposed a Bayesian hierarchical model to generate the match outcomes in terms of scores. Inference is conducted by MCMC. 
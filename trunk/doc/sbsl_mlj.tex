%%%%%%%%%%%%%%%%%%%%%%% file template.tex %%%%%%%%%%%%%%%%%%%%%%%%%
%
% This is a general template file for the LaTeX package SVJour3
% for Springer journals.          Springer Heidelberg 2010/09/16
%
% Copy it to a new file with a new name and use it as the basis
% for your article. Delete % signs as needed.
%
% This template includes a few options for different layouts and
% content for various journals. Please consult a previous issue of
% your journal as needed.
%
%%%%%%%%%%%%%%%%%%%%%%%%%%%%%%%%%%%%%%%%%%%%%%%%%%%%%%%%%%%%%%%%%%%
%
% First comes an example EPS file -- just ignore it and
% proceed on the \documentclass line
% your LaTeX will extract the file if required
%\begin{filecontents*}{example.eps}
%%!PS-Adobe-3.0 EPSF-3.0
%%%BoundingBox: 19 19 221 221
%%%CreationDate: Mon Sep 29 1997
%%%Creator: programmed by hand (JK)
%%%EndComments
%gsave
%newpath
%  20 20 moveto
%  20 220 lineto
%  220 220 lineto
%  220 20 lineto
%closepath
%2 setlinewidth
%gsave
%  .4 setgray fill
%grestore
%stroke
%grestore
%\end{filecontents*}
%
\RequirePackage{fix-cm}
%
%\documentclass{svjour3}                     % onecolumn (standard format)
%\documentclass[smallcondensed]{svjour3}     % onecolumn (ditto)
\documentclass[smallextended]{svjour3}       % onecolumn (second format)
%\documentclass[twocolumn]{svjour3}          % twocolumn
%
\smartqed  % flush right qed marks, e.g. at end of proof
%
% \newcommand{\keywords}[1]{\par\addvspace\baselineskip
% \noindent\keywordname\enspace\ignorespaces#1}

\usepackage{graphicx}
\usepackage{graphics}
\usepackage{amssymb}
\usepackage{amsmath}
%\usepackage{amsthm}
\usepackage{amsfonts}
\usepackage{amssymb}
\usepackage{subfigure}
\usepackage{epsfig}
\usepackage[hyphens]{url}
%\usepackage{hyperref}

\urldef{\mailsa}\path|{shengbo.guo}@xrce.xerox.com|
\urldef{\mailsb}\path|{scott.sanner, wray.buntine}@nicta.com.au|
\urldef{\mailsc}\path|{thore.graepel}@microsoft.com|
\newcommand{\unindent}{\hspace{-1mm}}
\newcommand{\unindentmore}{\hspace{-1.5mm}}
\def\argmax{\operatornamewithlimits{arg\,max}}
\def\argmin{\operatornamewithlimits{arg\,min}}
\long\def\COMMENT#1\ENDCOMMENT{\message{(Commented text...)}\par}
%
% \usepackage{mathptmx}      % use Times fonts if available on your TeX system
%
% insert here the call for the packages your document requires
%\usepackage{latexsym}
% etc.
%
% please place your own definitions here and don't use \def but
% \newcommand{}{}
%
% Insert the name of "your journal" with
% \journalname{myjournal}
%
\begin{document}

\title{Bayesian Score-Based Skill Learning}

% \subtitle{Do you have a subtitle?\\ If so, write it here}

%\titlerunning{Short form of title}        % if too long for running head

\author{Shengbo Guo \and Scott Sanner \and Thore Graepel \and Wray Buntine}
%\author{Unknown authors}
%
\authorrunning{S. Guo, S. Sanner, T. Graepel, W. Buntine}

%\institute{Xerox Research Centre Europe
%\and NICTA and the Australian National University
%\and Microsoft Research Cambridge}


\institute{Shengbo Guo \at
              Xerox Research Centre Europe \\
              % Tel.: +123-45-678910\\
              % Fax: +123-45-678910\\
              \email{shengbo.guo@xrce.xerox.com}           %  \\
%             \emph{Present address:} of F. Author  %  if needed
           \and
           Scott Sanner \at
              NICTA and ANU \\
              \email{scott.sanner@nicta.com.au}
           \and
           Thore Graepel \at
           		Microsoft Research Cambridge \\
           		\email{thore.graepel@microsoft.com}
           \and
           Wray Buntine  \at
              NICTA and ANU \\
              \email{wray.buntine@nicta.com.au}
}

\date{Received: date / Accepted: date}
% The correct dates will be entered by the editor


\maketitle

\begin{abstract}
\input abstract.tex
\keywords{variational inference, matchmaking, graphical models}

% \PACS{PACS code1 \and PACS code2 \and more}
% \subclass{MSC code1 \and MSC code2 \and more}
\end{abstract}

%%
\section{Introduction}
\input introduction.tex

%%
\section{Skill Learning using TrueSkill}
\input trueskill.tex

%%
\section{Score-based Bayesian Skill Models}
\input model.tex

%%
\section{Skill and Win Probability Inference}
\input inference.tex

%%
\section{Empirical Evaluation}
\input experiment.tex

%%
\section{Related Work}
\input relatedwork.tex

%%
\section{Conclusion}
\input conclusion.tex

\begin{acknowledgements}
We thank Marconi Barbosa, Guillaume Bouchard, David Stern and Onno Zoeter for interesting discussions, and we also thank the anonymous reviwers at ECML-PKDD'12 for their constructive comments, which help to improve the paper. NICTA is funded by the Australian Government as represented by the Department of Broadband, Communications and the Digital Economy and the Australian Research Council through the ICT Centre of Excellence program.
\end{acknowledgements}

% BibTeX users please use one of
%\bibliographystyle{spbasic}      % basic style, author-year citations
%\bibliographystyle{spmpsci}      % mathematics and physical sciences
%\bibliographystyle{spphys}       % APS-like style for physics
%\bibliography{}   % name your BibTeX data base

% Non-BibTeX users please use
\input bib.tex
\end{document}
% end of file template.tex


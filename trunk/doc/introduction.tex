In online gaming, it is important to pair players or teams of players
so as to optimise their gaming experience.  Game players often expect
competitors with comparable skills for the most enjoyable experience;
match experience can be compromised if one side consistently
outperforms the other.  \emph{Matchmaking} attempts to pair players
such that match results are close to being even or a draw.  Hence, a
prerequisite for good matchmaking is the ability to predict future
match results correctly from historical match outcomes --- a task
that is often cast in terms of latent skill learning.
%How to do
%this for matches with score-based outcomes is the question we address
%in this paper.

TrueSkill~\cite{herbrich06569} is a state-of-the-art Bayesian skill
learning system: it has been deployed in the Microsoft Xbox 360 online
gaming system for both matchmaking and player ranking.  For the case
of two teams/players, TrueSkill, like
Elo~\cite{elo78TheRatingOfChessPlayers}, is restricted to learn skills
from match outcomes in terms of win, lose, or draw (WLD).
While we conjecture that TrueSkill discards potentially
valuable skill information carried by score-based outcomes,
there are at least two arguments in favour of TrueSkill's WLD-based
skill learning approach:
\begin{itemize}
\item WLD-based systems can be applied to any game whose outcome space
is WLD, no matter what the underlying scoring system is.
\item In many games, the objective is not to win by the highest
score differential, but rather simply to win.  In this case,
it can be said that TrueSkill's skill modeling and learning from WLD
outcomes aligns well with the players' underlying objective.
\end{itemize}
%if the skill estimates of the system are revealed back to the
%players, there is a danger that the players may begin optimising their
%game play to influence their public skill estimate.  Hence, if the
%skill estimate is based on measurements other than the true game
%outcomes, this can lead to a distortion of the game's dynamics.
%However, if we assume that score information is available and
%estimated skill information is maintained privately, then it makes
%sense to attempt to exploit this additional score information in
On the other hand, we note that discarding score results
ignores two important sources of information:
\begin{itemize}
\item High (or low) score
differentials can provide insight into relative team strengths.
\item Two dimensional score outcomes (i.e., a score for each side) provide a
direct basis for inferring separate offense and defense strengths for
each team, hence permitting finer-grained modeling of performance against
future opponents.
%Peeking ahead, we note that the predictive
%improvements of score-based models over TrueSkill as shown in
%Section~\ref{sec:results} support this conjecture in many cases.
\end{itemize}

In this work, we augment the TrueSkill model of WLD skill learning to
learn from score-based outcomes.  We explore single skill models as
well as separate offense/defense skill models made possible via
score-based modeling. We also investigate both Gaussian and Poisson
score likelihood models, deriving a novel variational update for
approximate Bayesian inference in the latter case.  We evaluate these
novel Bayesian score-based skill-learning models in comparison to
TrueSkill (for WLD outcomes) on three datasets: 14 years of match
outcomes for the UK Premier League, 11 years of match outcomes for the
Australian Football (Rugby) League (AFL), and three days covering
6,000+ online match outcomes in the Halo 2 XBox video game.  Empirical
evaluations demonstrate that the new score-based models (a) provide
more accurate win/loss probability estimates than TrueSkill (in terms
of information gain) with limited amounts of training data, (b)
provide competitive and often better win/loss classification
performance than TrueSkill (in terms of area under the curve), and (c)
provide reasonably accurate score predictions with an appropriate
likelihood --- prediction for which TrueSkill was not designed but
important in cases such as tournaments that rank (or break ties) by
points, professional sports betting and bookmaking, and game-play
strategy decisions that are dependent on final score projections.